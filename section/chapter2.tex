\chapter{Mapserver}
\section{Instalasi Mapserver}
\subsection{Langkah-langkah instalasi Mapserver}
\begin{enumerate}
    \item Download Aplikasi Map Server atau MS4W, pilih ms4w-3.2.8-setup.exe.
    \item Setelah selesai download, pili file yang terdownload kemudian klik kanan, pilih Run Administrator. Maka akan muncul seperti gambar \ref{instal1}, klik double I Agree.
    \begin{figure}[!htbp]
    \centering
    \includegraphics[scale=0.5]{figures/instal1.jpg}
    \label{instal1}
\end{figure}
    \item Selanjutnya pilih MapServer Itasca Demo Application, dan jangan ubah yang lainnya, kemudian pilih Next > \ref{instal2}.
    \begin{figure}[!htbp]
    \centering
    \includegraphics[scale=0.5]{figures/instal2.jpg}
    \label{instal2}
\end{figure}
    \item Setelah itu pilih penyimpanan nya dan saya akan menyimpannya di C:\ kemudian pilih Next> \ref{instal3}. 
    \begin{figure}[!htbp]
    \centering
    \includegraphics[scale=0.5]{figures/instal3.jpg}
    \label{instal3}
\end{figure}
    \item Masukkan port 82 Pada Apache Port kemudian pilih Install, seperti gambar \ref{instal4}
     \begin{figure}[!htbp]
    \centering
    \includegraphics[scale=0.5]{figures/instal4.jpg}
    \label{instal4}
\end{figure}
    \item Tunggu proses instalisasi nya sepreti gambar \ref{instal5}. 
    \begin{figure}[!htbp]
    \centering
    \includegraphics[scale=0.5]{figures/instal5.jpg}
    \label{instal5}
\end{figure}
    \item Setelah instalasi nya selesai akan muncul seperti ini, kemudian tunggu lagi prosesnya sampai Complete seperti gambar \ref{instal6}.
    \begin{figure}[!htbp]
    \centering
    \includegraphics[scale=0.5]{figures/instal6.jpg}
    \label{instal6}
\end{figure}    
    \item Pada pencarian masukkan localhost:82 dimana 82 merupakan port yang telah di tetapkan pada proses instalasi Map Server nya, seperti pada gambar \ref{instal7}. 
    \begin{figure}[!htbp]
    \centering
    \includegraphics[scale=0.4]{figures/instal7.jpg}
    \label{instal7}
\end{figure}  

\end{enumerate}

